% Options for packages loaded elsewhere
\PassOptionsToPackage{unicode}{hyperref}
\PassOptionsToPackage{hyphens}{url}
\documentclass[
]{article}
\usepackage{xcolor}
\usepackage{amsmath,amssymb}
\setcounter{secnumdepth}{-\maxdimen} % remove section numbering
\usepackage{iftex}
\ifPDFTeX
  \usepackage[T1]{fontenc}
  \usepackage[utf8]{inputenc}
  \usepackage{textcomp} % provide euro and other symbols
\else % if luatex or xetex
  \usepackage{unicode-math} % this also loads fontspec
  \defaultfontfeatures{Scale=MatchLowercase}
  \defaultfontfeatures[\rmfamily]{Ligatures=TeX,Scale=1}
\fi
\usepackage{lmodern}
\ifPDFTeX\else
  % xetex/luatex font selection
\fi
% Use upquote if available, for straight quotes in verbatim environments
\IfFileExists{upquote.sty}{\usepackage{upquote}}{}
\IfFileExists{microtype.sty}{% use microtype if available
  \usepackage[]{microtype}
  \UseMicrotypeSet[protrusion]{basicmath} % disable protrusion for tt fonts
}{}
\makeatletter
\@ifundefined{KOMAClassName}{% if non-KOMA class
  \IfFileExists{parskip.sty}{%
    \usepackage{parskip}
  }{% else
    \setlength{\parindent}{0pt}
    \setlength{\parskip}{6pt plus 2pt minus 1pt}}
}{% if KOMA class
  \KOMAoptions{parskip=half}}
\makeatother
\usepackage{color}
\usepackage{fancyvrb}
\newcommand{\VerbBar}{|}
\newcommand{\VERB}{\Verb[commandchars=\\\{\}]}
\DefineVerbatimEnvironment{Highlighting}{Verbatim}{commandchars=\\\{\}}
% Add ',fontsize=\small' for more characters per line
\newenvironment{Shaded}{}{}
\newcommand{\AlertTok}[1]{\textcolor[rgb]{1.00,0.00,0.00}{\textbf{#1}}}
\newcommand{\AnnotationTok}[1]{\textcolor[rgb]{0.38,0.63,0.69}{\textbf{\textit{#1}}}}
\newcommand{\AttributeTok}[1]{\textcolor[rgb]{0.49,0.56,0.16}{#1}}
\newcommand{\BaseNTok}[1]{\textcolor[rgb]{0.25,0.63,0.44}{#1}}
\newcommand{\BuiltInTok}[1]{\textcolor[rgb]{0.00,0.50,0.00}{#1}}
\newcommand{\CharTok}[1]{\textcolor[rgb]{0.25,0.44,0.63}{#1}}
\newcommand{\CommentTok}[1]{\textcolor[rgb]{0.38,0.63,0.69}{\textit{#1}}}
\newcommand{\CommentVarTok}[1]{\textcolor[rgb]{0.38,0.63,0.69}{\textbf{\textit{#1}}}}
\newcommand{\ConstantTok}[1]{\textcolor[rgb]{0.53,0.00,0.00}{#1}}
\newcommand{\ControlFlowTok}[1]{\textcolor[rgb]{0.00,0.44,0.13}{\textbf{#1}}}
\newcommand{\DataTypeTok}[1]{\textcolor[rgb]{0.56,0.13,0.00}{#1}}
\newcommand{\DecValTok}[1]{\textcolor[rgb]{0.25,0.63,0.44}{#1}}
\newcommand{\DocumentationTok}[1]{\textcolor[rgb]{0.73,0.13,0.13}{\textit{#1}}}
\newcommand{\ErrorTok}[1]{\textcolor[rgb]{1.00,0.00,0.00}{\textbf{#1}}}
\newcommand{\ExtensionTok}[1]{#1}
\newcommand{\FloatTok}[1]{\textcolor[rgb]{0.25,0.63,0.44}{#1}}
\newcommand{\FunctionTok}[1]{\textcolor[rgb]{0.02,0.16,0.49}{#1}}
\newcommand{\ImportTok}[1]{\textcolor[rgb]{0.00,0.50,0.00}{\textbf{#1}}}
\newcommand{\InformationTok}[1]{\textcolor[rgb]{0.38,0.63,0.69}{\textbf{\textit{#1}}}}
\newcommand{\KeywordTok}[1]{\textcolor[rgb]{0.00,0.44,0.13}{\textbf{#1}}}
\newcommand{\NormalTok}[1]{#1}
\newcommand{\OperatorTok}[1]{\textcolor[rgb]{0.40,0.40,0.40}{#1}}
\newcommand{\OtherTok}[1]{\textcolor[rgb]{0.00,0.44,0.13}{#1}}
\newcommand{\PreprocessorTok}[1]{\textcolor[rgb]{0.74,0.48,0.00}{#1}}
\newcommand{\RegionMarkerTok}[1]{#1}
\newcommand{\SpecialCharTok}[1]{\textcolor[rgb]{0.25,0.44,0.63}{#1}}
\newcommand{\SpecialStringTok}[1]{\textcolor[rgb]{0.73,0.40,0.53}{#1}}
\newcommand{\StringTok}[1]{\textcolor[rgb]{0.25,0.44,0.63}{#1}}
\newcommand{\VariableTok}[1]{\textcolor[rgb]{0.10,0.09,0.49}{#1}}
\newcommand{\VerbatimStringTok}[1]{\textcolor[rgb]{0.25,0.44,0.63}{#1}}
\newcommand{\WarningTok}[1]{\textcolor[rgb]{0.38,0.63,0.69}{\textbf{\textit{#1}}}}
\setlength{\emergencystretch}{3em} % prevent overfull lines
\providecommand{\tightlist}{%
  \setlength{\itemsep}{0pt}\setlength{\parskip}{0pt}}
\usepackage{bookmark}
\IfFileExists{xurl.sty}{\usepackage{xurl}}{} % add URL line breaks if available
\urlstyle{same}
\hypersetup{
  hidelinks,
  pdfcreator={LaTeX via pandoc}}

\author{}
\date{}

\begin{document}

\section{Analyse du projet CESIzen}\label{analyse-du-projet-cesizen}

\subsection{1. Analyse fonctionnelle générale
(UML)}\label{analyse-fonctionnelle-guxe9nuxe9rale-uml}

\subsubsection{1.1 Diagramme de cas d’utilisation
global}\label{diagramme-de-cas-dutilisation-global}

\begin{Shaded}
\begin{Highlighting}[]
\NormalTok{graph TD}
\NormalTok{    \%\% Acteurs}
\NormalTok{    Guest((Utilisateur non authentifié))}
\NormalTok{    User((Utilisateur authentifié))}
\NormalTok{    Admin((Administrateur))}
    
\NormalTok{    \%\% Cas d\textquotesingle{}utilisation {-} Général}
\NormalTok{    Register[S\textquotesingle{}inscrire]}
\NormalTok{    Login[Se connecter]}
\NormalTok{    ResetPassword[Réinitialiser mot de passe]}
    
\NormalTok{    \%\% Cas d\textquotesingle{}utilisation {-} Fonctionnalités}
\NormalTok{    ViewZenGarden[Visualiser le jardin zen]}
\NormalTok{    ExploreBreathing[Explorer les exercices de respiration]}
\NormalTok{    TakeStressTest[Effectuer un test de stress]}
\NormalTok{    ViewDashboard[Consulter le tableau de bord]}
\NormalTok{    TrackEmotions[Suivre ses émotions]}
\NormalTok{    ViewActivities[Consulter les activités]}
\NormalTok{    AddFavorite[Ajouter une activité aux favoris]}
\NormalTok{    ViewFAQ[Consulter la FAQ]}
    
\NormalTok{    \%\% Cas d\textquotesingle{}utilisation {-} Admin}
\NormalTok{    ManageUsers[Gérer les utilisateurs]}
\NormalTok{    ManageStressQuestions[Gérer les questions de stress]}
\NormalTok{    ManageEmotionTypes[Gérer les types d\textquotesingle{}émotions]}
\NormalTok{    ManageActivities[Gérer les activités]}
\NormalTok{    ManageContent[Gérer le contenu des pages]}
    
\NormalTok{    \%\% Relations {-} Guest}
\NormalTok{    Guest {-}{-}\textgreater{} Register}
\NormalTok{    Guest {-}{-}\textgreater{} Login}
\NormalTok{    Guest {-}{-}\textgreater{} ResetPassword}
\NormalTok{    Guest {-}{-}\textgreater{} ViewZenGarden}
\NormalTok{    Guest {-}{-}\textgreater{} ViewFAQ}
    
\NormalTok{    \%\% Relations {-} User}
\NormalTok{    User {-}{-}\textgreater{} ExploreBreathing}
\NormalTok{    User {-}{-}\textgreater{} TakeStressTest}
\NormalTok{    User {-}{-}\textgreater{} ViewDashboard}
\NormalTok{    User {-}{-}\textgreater{} TrackEmotions}
\NormalTok{    User {-}{-}\textgreater{} ViewActivities}
\NormalTok{    User {-}{-}\textgreater{} AddFavorite}
    
\NormalTok{    \%\% Relations {-} Admin}
\NormalTok{    Admin {-}{-}\textgreater{} ManageUsers}
\NormalTok{    Admin {-}{-}\textgreater{} ManageStressQuestions}
\NormalTok{    Admin {-}{-}\textgreater{} ManageEmotionTypes}
\NormalTok{    Admin {-}{-}\textgreater{} ManageActivities}
\NormalTok{    Admin {-}{-}\textgreater{} ManageContent}
    
\NormalTok{    \%\% Héritage}
\NormalTok{    Admin {-}{-}\textgreater{} User}
\end{Highlighting}
\end{Shaded}

\begin{quote}
\textbf{Légende du diagramme} : Ce diagramme de cas d’utilisation
représente les interactions entre les différents types d’utilisateurs et
les fonctionnalités du système CESIzen. Les cercles représentent les
acteurs et les rectangles représentent les fonctionnalités accessibles.
La flèche entre Admin et User indique que l’administrateur hérite de
toutes les fonctionnalités d’un utilisateur standard.
\end{quote}

\subsubsection{1.2 Diagramme de classes}\label{diagramme-de-classes}

\begin{Shaded}
\begin{Highlighting}[]
\NormalTok{classDiagram}
\NormalTok{    User {-}{-}\textgreater{} StressResult}
\NormalTok{    User {-}{-}\textgreater{} Emotion}
\NormalTok{    User {-}{-}\textgreater{} Favorite}
\NormalTok{    EmotionType {-}{-}\textgreater{} Emotion}
\NormalTok{    EmotionType o{-}{-} EmotionType : "parent"}
\NormalTok{    Activity {-}{-}\textgreater{} Favorite}
    
\NormalTok{    class User \{}
\NormalTok{        +int id}
\NormalTok{        +string name}
\NormalTok{        +string email}
\NormalTok{        +string password}
\NormalTok{        +Role role}
\NormalTok{        +datetime createdAt}
\NormalTok{        +datetime updatedAt}
\NormalTok{        +bool isActive}
\NormalTok{        +string resetToken}
\NormalTok{        +datetime resetTokenExpires}
\NormalTok{    \}}
    
\NormalTok{    class StressQuestion \{}
\NormalTok{        +int id}
\NormalTok{        +string event}
\NormalTok{        +int points}
\NormalTok{    \}}
    
\NormalTok{    class StressResult \{}
\NormalTok{        +int id}
\NormalTok{        +int userId}
\NormalTok{        +int totalScore}
\NormalTok{        +datetime createdAt}
\NormalTok{    \}}
    
\NormalTok{    class Emotion \{}
\NormalTok{        +int id}
\NormalTok{        +int userId}
\NormalTok{        +int emotionId}
\NormalTok{        +datetime date}
\NormalTok{        +string comment}
\NormalTok{    \}}
    
\NormalTok{    class EmotionType \{}
\NormalTok{        +int id}
\NormalTok{        +string name}
\NormalTok{        +int level}
\NormalTok{        +int parentId}
\NormalTok{        +string color}
\NormalTok{        +string bgColor}
\NormalTok{    \}}
    
\NormalTok{    class Activity \{}
\NormalTok{        +int id}
\NormalTok{        +string title}
\NormalTok{        +string description}
\NormalTok{        +ActivityCategory category}
\NormalTok{        +string duration}
\NormalTok{        +string level}
\NormalTok{        +string location}
\NormalTok{        +string equipment}
\NormalTok{        +bool isActive}
\NormalTok{        +datetime createdAt}
\NormalTok{        +datetime updatedAt}
\NormalTok{    \}}
    
\NormalTok{    class Favorite \{}
\NormalTok{        +int id}
\NormalTok{        +int userId}
\NormalTok{        +int activityId}
\NormalTok{        +datetime createdAt}
\NormalTok{    \}}
    
\NormalTok{    class PageContent \{}
\NormalTok{        +int id}
\NormalTok{        +string page}
\NormalTok{        +string title}
\NormalTok{        +string content}
\NormalTok{        +datetime updatedAt}
\NormalTok{    \}}
    
\NormalTok{    class Role \{}
\NormalTok{        \textless{}\textless{}enumeration\textgreater{}\textgreater{}}
\NormalTok{        USER}
\NormalTok{        ADMIN}
\NormalTok{    \}}
    
\NormalTok{    class ActivityCategory \{}
\NormalTok{        \textless{}\textless{}enumeration\textgreater{}\textgreater{}}
\NormalTok{        BIEN\_ETRE}
\NormalTok{        SPORT}
\NormalTok{        ART}
\NormalTok{        AVENTURE}
\NormalTok{    \}}
\end{Highlighting}
\end{Shaded}

\begin{quote}
\textbf{Légende du diagramme} : Ce diagramme de classes présente le
modèle de données de l’application CESIzen. Il montre les entités
principales (User, Activity, Emotion, etc.) avec leurs attributs et les
relations entre elles. La notation “1” – “0..*” indique une relation
un-à-plusieurs, où par exemple un utilisateur peut posséder plusieurs
résultats de test de stress.
\end{quote}

\subsubsection{1.3 Diagramme de séquence pour le test de
stress}\label{diagramme-de-suxe9quence-pour-le-test-de-stress}

\begin{Shaded}
\begin{Highlighting}[]
\NormalTok{sequenceDiagram}
\NormalTok{    actor User}
\NormalTok{    participant Page as Page Test Stress}
\NormalTok{    participant API}
\NormalTok{    participant DB as Base de données}
    
\NormalTok{    User{-}\textgreater{}\textgreater{}Page: Accède à la page de test}
\NormalTok{    Page{-}\textgreater{}\textgreater{}API: GET /api/stress/questions}
\NormalTok{    API{-}\textgreater{}\textgreater{}DB: Récupère les questions}
\NormalTok{    DB{-}{-}\textgreater{}\textgreater{}API: Retourne les questions}
\NormalTok{    API{-}{-}\textgreater{}\textgreater{}Page: Renvoie les questions}
\NormalTok{    Page{-}{-}\textgreater{}\textgreater{}User: Affiche les questions}
    
\NormalTok{    User{-}\textgreater{}\textgreater{}Page: Sélectionne les événements stressants}
\NormalTok{    User{-}\textgreater{}\textgreater{}Page: Clique sur "Voir mon résultat"}
\NormalTok{    Page{-}\textgreater{}\textgreater{}API: POST /api/stress/submit}
\NormalTok{    API{-}\textgreater{}\textgreater{}DB: Enregistre le résultat du test}
\NormalTok{    API{-}\textgreater{}\textgreater{}API: Calcule le score de stress}
\NormalTok{    API{-}{-}\textgreater{}\textgreater{}Page: Renvoie le score et l\textquotesingle{}évaluation}
\NormalTok{    Page{-}{-}\textgreater{}\textgreater{}User: Affiche le résultat et les recommandations}
    
\NormalTok{    \%\% Légende avec note explicative}
\NormalTok{    note over User,DB: Flux d\textquotesingle{}interaction pour le test de stress}
\end{Highlighting}
\end{Shaded}

\begin{quote}
\textbf{Légende du diagramme} : Ce diagramme de séquence illustre les
interactions entre l’utilisateur, l’interface, l’API et la base de
données lors de la réalisation d’un test de stress. Il montre
chronologiquement comment les données circulent depuis la demande de
questions jusqu’à l’affichage du résultat final à l’utilisateur.
\end{quote}

\subsection{2. Analyse fonctionnelle
détaillée}\label{analyse-fonctionnelle-duxe9tailluxe9e}

\subsubsection{2.1 Module
d’authentification}\label{module-dauthentification}

\paragraph{Fonctionnalités}\label{fonctionnalituxe9s}

\begin{itemize}
\tightlist
\item
  Inscription d’un nouvel utilisateur
\item
  Connexion d’un utilisateur existant
\item
  Réinitialisation du mot de passe
\item
  Gestion des sessions utilisateurs
\item
  Séparation des rôles (utilisateur standard, administrateur)
\end{itemize}

\paragraph{Règles de gestion}\label{ruxe8gles-de-gestion}

\begin{itemize}
\tightlist
\item
  Email unique pour chaque utilisateur
\item
  Mot de passe sécurisé et haché
\item
  Session expirée après une période d’inactivité
\item
  Accès aux fonctionnalités administratives limité aux utilisateurs
  ayant le rôle ADMIN
\end{itemize}

\subsubsection{2.2 Module de respiration}\label{module-de-respiration}

\paragraph{Fonctionnalités}\label{fonctionnalituxe9s-1}

\begin{itemize}
\tightlist
\item
  Sélection d’exercices de respiration prédéfinis
\item
  Guidage visuel pour les phases d’inspiration, de rétention et
  d’expiration
\item
  Paramétrage des durées pour chaque phase
\item
  Suivi de session (durée, exercices effectués)
\end{itemize}

\paragraph{Règles de gestion}\label{ruxe8gles-de-gestion-1}

\begin{itemize}
\tightlist
\item
  Validation des durées de phase (valeurs positives et cohérentes)
\item
  Adaptation de l’interface aux différentes tailles d’écran
\item
  Animation fluide pour guider l’utilisateur
\end{itemize}

\begin{Shaded}
\begin{Highlighting}[]
\NormalTok{flowchart TD}
\NormalTok{    A[Utilisateur] {-}{-}\textgreater{} B\{Sélection d\textquotesingle{}exercice\}}
\NormalTok{    B {-}{-}\textgreater{} C[Respiration 4{-}7{-}8]}
\NormalTok{    B {-}{-}\textgreater{} D[Respiration 5{-}5]}
\NormalTok{    B {-}{-}\textgreater{} E[Respiration 4{-}6]}
    
\NormalTok{    C {-}{-}\textgreater{} F[Démarrer]}
\NormalTok{    D {-}{-}\textgreater{} F}
\NormalTok{    E {-}{-}\textgreater{} F}
    
\NormalTok{    F {-}{-}\textgreater{} G\{Phase\}}
\NormalTok{    G {-}{-}\textgreater{}|Inspiration| H[Cercle s\textquotesingle{}agrandit]}
\NormalTok{    G {-}{-}\textgreater{}|Rétention| I[Cercle stable]}
\NormalTok{    G {-}{-}\textgreater{}|Expiration| J[Cercle rétrécit]}
    
\NormalTok{    H {-}{-}\textgreater{} G}
\NormalTok{    I {-}{-}\textgreater{} G}
\NormalTok{    J {-}{-}\textgreater{} G}
    
\NormalTok{    F {-}{-}\textgreater{} K[Arrêter]}
\end{Highlighting}
\end{Shaded}

\begin{quote}
\textbf{Légende du diagramme} : Ce diagramme illustre le flux
d’utilisation du module de respiration. L’utilisateur sélectionne
d’abord un exercice de respiration, puis démarre la séance. Le système
guide ensuite l’utilisateur à travers les différentes phases
(inspiration, rétention, expiration) avec des animations visuelles
correspondantes.
\end{quote}

\subsubsection{2.3 Module de tracker de
stress}\label{module-de-tracker-de-stress}

\paragraph{Fonctionnalités}\label{fonctionnalituxe9s-2}

\begin{itemize}
\tightlist
\item
  Test de stress basé sur des événements de vie (questionnaire)
\item
  Calcul du score de stress
\item
  Historique des tests effectués
\item
  Recommandations basées sur le niveau de stress
\end{itemize}

\paragraph{Règles de gestion}\label{ruxe8gles-de-gestion-2}

\begin{itemize}
\tightlist
\item
  Sélection multiple d’événements stressants
\item
  Algorithme de calcul du score basé sur la somme des points de chaque
  événement
\item
  Catégorisation du niveau de stress (faible, modéré, élevé)
\item
  Stockage sécurisé des résultats associés à l’utilisateur connecté
\end{itemize}

\begin{Shaded}
\begin{Highlighting}[]
\NormalTok{stateDiagram{-}v2}
\NormalTok{    [*] {-}{-}\textgreater{} Étape1}
\NormalTok{    Étape1 {-}{-}\textgreater{} Étape2 : Suivant}
\NormalTok{    Étape2 {-}{-}\textgreater{} Étape3 : Suivant}
\NormalTok{    Étape3 {-}{-}\textgreater{} Étape4 : Suivant}
\NormalTok{    Étape4 {-}{-}\textgreater{} Calcul : Voir mon résultat}
    
\NormalTok{    Étape2 {-}{-}\textgreater{} Étape1 : Précédent}
\NormalTok{    Étape3 {-}{-}\textgreater{} Étape2 : Précédent}
\NormalTok{    Étape4 {-}{-}\textgreater{} Étape3 : Précédent}
    
\NormalTok{    Calcul {-}{-}\textgreater{} ResultatFaible : Score \textless{} 150}
\NormalTok{    Calcul {-}{-}\textgreater{} ResultatModéré : 150 ≤ Score \textless{} 300}
\NormalTok{    Calcul {-}{-}\textgreater{} ResultatÉlevé : Score ≥ 300}
    
\NormalTok{    ResultatFaible {-}{-}\textgreater{} [*]}
\NormalTok{    ResultatModéré {-}{-}\textgreater{} [*]}
\NormalTok{    ResultatÉlevé {-}{-}\textgreater{} [*]}
    
\NormalTok{    note right of Étape1 : Questions 1{-}10}
\NormalTok{    note right of Étape2 : Questions 11{-}20}
\NormalTok{    note right of Étape3 : Questions 21{-}30}
\NormalTok{    note right of Étape4 : Questions 31{-}40}
    
\NormalTok{    note left of ResultatFaible : Niveau de stress gérable}
\NormalTok{    note left of ResultatModéré : Stress significatif}
\NormalTok{    note left of ResultatÉlevé : Stress élevé {-} Consulter un professionnel}
\end{Highlighting}
\end{Shaded}

\begin{quote}
\textbf{Légende du diagramme} : Ce diagramme d’états représente le
processus de test de stress dans l’application. L’utilisateur progresse
à travers quatre étapes de questions, puis obtient un résultat classé
selon le score total calculé. Les trois catégories de résultat (faible,
modéré, élevé) déclenchent différentes recommandations adaptées au
niveau de stress détecté.
\end{quote}

\subsubsection{2.4 Module de tracker
d’émotions}\label{module-de-tracker-duxe9motions}

\paragraph{Fonctionnalités}\label{fonctionnalituxe9s-3}

\begin{itemize}
\tightlist
\item
  Enregistrement des émotions ressenties
\item
  Catégorisation des émotions (émotions principales et sous-émotions)
\item
  Historique des émotions enregistrées
\item
  Visualisation des tendances émotionnelles
\end{itemize}

\paragraph{Règles de gestion}\label{ruxe8gles-de-gestion-3}

\begin{itemize}
\tightlist
\item
  Structure hiérarchique des émotions (niveau 1: émotions principales,
  niveau 2: sous-émotions)
\item
  Chaque enregistrement peut être commenté
\item
  Affichage chronologique des émotions
\item
  Regroupement par période (jour, semaine, mois)
\end{itemize}

\begin{Shaded}
\begin{Highlighting}[]
\NormalTok{flowchart TD}
\NormalTok{    A[Utilisateur] {-}{-}\textgreater{} B\{Sélection émotion principale\}}
\NormalTok{    B {-}{-}\textgreater{} Joie}
\NormalTok{    B {-}{-}\textgreater{} Tristesse}
\NormalTok{    B {-}{-}\textgreater{} Colère}
\NormalTok{    B {-}{-}\textgreater{} Peur}
\NormalTok{    B {-}{-}\textgreater{} Surprise}
\NormalTok{    B {-}{-}\textgreater{} Dégoût}
    
\NormalTok{    Joie {-}{-}\textgreater{} SousJoie\{Sous{-}émotions\}}
\NormalTok{    SousJoie {-}{-}\textgreater{} Contentement}
\NormalTok{    SousJoie {-}{-}\textgreater{} Fierté}
\NormalTok{    SousJoie {-}{-}\textgreater{} Amusement}
    
\NormalTok{    Tristesse {-}{-}\textgreater{} SousTristesse\{Sous{-}émotions\}}
\NormalTok{    SousTristesse {-}{-}\textgreater{} Chagrin}
\NormalTok{    SousTristesse {-}{-}\textgreater{} Solitude}
\NormalTok{    SousTristesse {-}{-}\textgreater{} Désespoir}
    
\NormalTok{    Colère {-}{-}\textgreater{} SousColère\{Sous{-}émotions\}}
\NormalTok{    SousColère {-}{-}\textgreater{} Irritation}
\NormalTok{    SousColère {-}{-}\textgreater{} Fureur}
\NormalTok{    SousColère {-}{-}\textgreater{} Rancœur}
    
\NormalTok{    Contentement {-}{-}\textgreater{} C[Ajout commentaire]}
\NormalTok{    Chagrin {-}{-}\textgreater{} C}
\NormalTok{    Irritation {-}{-}\textgreater{} C}
    
\NormalTok{    C {-}{-}\textgreater{} D[Enregistrement]}
\NormalTok{    D {-}{-}\textgreater{} E[Historique des émotions]}
\end{Highlighting}
\end{Shaded}

\begin{quote}
\textbf{Légende du diagramme} : Ce diagramme illustre le processus
d’enregistrement des émotions dans l’application. L’utilisateur
sélectionne d’abord une émotion principale (joie, tristesse, colère,
etc.), puis une sous-émotion plus précise. Il peut ensuite ajouter un
commentaire personnel pour contextualiser son ressenti avant
d’enregistrer l’entrée dans son historique d’émotions.
\end{quote}

\subsubsection{2.5 Module d’activités}\label{module-dactivituxe9s}

\paragraph{Fonctionnalités}\label{fonctionnalituxe9s-4}

\begin{itemize}
\tightlist
\item
  Catalogue d’activités de bien-être
\item
  Filtrage par catégorie, durée et niveau
\item
  Ajout d’activités aux favoris
\item
  Détails complets sur chaque activité
\end{itemize}

\paragraph{Règles de gestion}\label{ruxe8gles-de-gestion-4}

\begin{itemize}
\tightlist
\item
  Catégorisation des activités (BIEN\_ETRE, SPORT, ART, AVENTURE)
\item
  Gestion de la disponibilité des activités (isActive)
\item
  Limitation des favoris (pas de doublons)
\item
  Affichage adapté selon la disponibilité des équipements
\end{itemize}

\begin{Shaded}
\begin{Highlighting}[]
\NormalTok{flowchart LR}
\NormalTok{    A[Utilisateur] {-}{-}\textgreater{} B[Catalogue d\textquotesingle{}activités]}
\NormalTok{    B {-}{-}\textgreater{} C\{Filtres\}}
    
\NormalTok{    C {-}{-}\textgreater{}|Catégorie| D[BIEN\_ETRE]}
\NormalTok{    C {-}{-}\textgreater{}|Catégorie| E[SPORT]}
\NormalTok{    C {-}{-}\textgreater{}|Catégorie| F[ART]}
\NormalTok{    C {-}{-}\textgreater{}|Catégorie| G[AVENTURE]}
    
\NormalTok{    C {-}{-}\textgreater{}|Durée| H[Court \textless{} 30min]}
\NormalTok{    C {-}{-}\textgreater{}|Durée| I[Moyen 30min{-}1h]}
\NormalTok{    C {-}{-}\textgreater{}|Durée| J[Long \textgreater{} 1h]}
    
\NormalTok{    C {-}{-}\textgreater{}|Niveau| K[Débutant]}
\NormalTok{    C {-}{-}\textgreater{}|Niveau| L[Intermédiaire]}
\NormalTok{    C {-}{-}\textgreater{}|Niveau| M[Avancé]}
    
\NormalTok{    D {-}{-}\textgreater{} N[Résultats filtrés]}
\NormalTok{    E {-}{-}\textgreater{} N}
\NormalTok{    F {-}{-}\textgreater{} N}
\NormalTok{    G {-}{-}\textgreater{} N}
\NormalTok{    H {-}{-}\textgreater{} N}
\NormalTok{    I {-}{-}\textgreater{} N}
\NormalTok{    J {-}{-}\textgreater{} N}
\NormalTok{    K {-}{-}\textgreater{} N}
\NormalTok{    L {-}{-}\textgreater{} N}
\NormalTok{    M {-}{-}\textgreater{} N}
    
\NormalTok{    N {-}{-}\textgreater{} O[Détails activité]}
\NormalTok{    O {-}{-}\textgreater{} P\{Actions\}}
\NormalTok{    P {-}{-}\textgreater{}|Ajouter aux favoris| Q[Favoris]}
\NormalTok{    P {-}{-}\textgreater{}|Retirer des favoris| Q}
\end{Highlighting}
\end{Shaded}

\begin{quote}
\textbf{Légende du diagramme} : Ce diagramme illustre le processus de
navigation et d’interaction avec le catalogue d’activités. L’utilisateur
peut filtrer les activités selon trois critères principaux (catégorie,
durée, niveau de difficulté), consulter les détails d’une activité
spécifique et gérer ses favoris.
\end{quote}

\subsubsection{2.6 Module du jardin zen}\label{module-du-jardin-zen}

\paragraph{Fonctionnalités}\label{fonctionnalituxe9s-5}

\begin{itemize}
\tightlist
\item
  Interface interactive représentant un jardin zen
\item
  Éléments interactifs (bassin, lanterne, pierre, pont)
\item
  Ambiance sonore relaxante
\item
  Navigation intuitive vers les fonctionnalités de l’application
\end{itemize}

\paragraph{Règles de gestion}\label{ruxe8gles-de-gestion-5}

\begin{itemize}
\tightlist
\item
  Animations fluides et non-intrusives
\item
  Réactivité aux interactions utilisateur
\item
  Adaptation à différentes tailles d’écran
\item
  Sons activables/désactivables
\end{itemize}

\begin{Shaded}
\begin{Highlighting}[]
\NormalTok{flowchart TD}
\NormalTok{    A[Utilisateur] {-}{-}\textgreater{} B[Jardin Zen Interface]}
\NormalTok{    B {-}{-}\textgreater{} C\{Éléments interactifs\}}
    
\NormalTok{    C {-}{-}\textgreater{} D[Bassin]}
\NormalTok{    C {-}{-}\textgreater{} E[Lanterne]}
\NormalTok{    C {-}{-}\textgreater{} F[Pierre]}
\NormalTok{    C {-}{-}\textgreater{} G[Pont]}
\NormalTok{    C {-}{-}\textgreater{} H[Ambiance sonore]}
    
\NormalTok{    D {-}{-}\textgreater{}|Interaction| I[Respiration]}
\NormalTok{    E {-}{-}\textgreater{}|Interaction| J[Activités]}
\NormalTok{    F {-}{-}\textgreater{}|Interaction| K[Tracker de stress]}
\NormalTok{    G {-}{-}\textgreater{}|Interaction| L[Tracker d\textquotesingle{}émotions]}
\NormalTok{    H {-}{-}\textgreater{}|Toggle| M[Son ON/OFF]}
\end{Highlighting}
\end{Shaded}

\begin{quote}
\textbf{Légende du diagramme} : Ce diagramme illustre l’interface du
jardin zen et ses éléments interactifs. Chaque élément (bassin,
lanterne, pierre, pont) sert de point d’entrée naturel vers une
fonctionnalité spécifique de l’application, créant ainsi une navigation
intuitive et immersive. Le contrôle d’ambiance sonore permet de
personnaliser l’expérience utilisateur.
\end{quote}

\subsubsection{2.7 Module de profil
utilisateur}\label{module-de-profil-utilisateur}

\paragraph{Fonctionnalités}\label{fonctionnalituxe9s-6}

\begin{itemize}
\tightlist
\item
  Affichage des informations du profil
\item
  Modification des informations personnelles
\item
  Visualisation des activités favorites
\item
  Historique des tests de stress et des émotions
\end{itemize}

\paragraph{Règles de gestion}\label{ruxe8gles-de-gestion-6}

\begin{itemize}
\tightlist
\item
  Validation des données saisies
\item
  Protection des données personnelles
\item
  Confirmation pour les modifications sensibles
\item
  Limitation de l’accès aux seuls utilisateurs concernés
\end{itemize}

\subsection{3. Analyse technique}\label{analyse-technique}

\subsubsection{3.1 Architecture du projet}\label{architecture-du-projet}

\begin{Shaded}
\begin{Highlighting}[]
\NormalTok{flowchart TD}
\NormalTok{    \%\% Composants principaux avec couleurs}
\NormalTok{    A[Frontend] {-}{-}\textgreater{} B[API Routes]}
\NormalTok{    B {-}{-}\textgreater{} C[Services]}
\NormalTok{    C {-}{-}\textgreater{} D[Prisma ORM]}
\NormalTok{    D {-}{-}\textgreater{} E[(Base de données MySQL)]}
    
\NormalTok{    \%\% Sous{-}structures}
\NormalTok{    subgraph Frontend["Frontend (Next.js)"]}
\NormalTok{        F["Pages (.tsx)"]}
\NormalTok{        G["Components (.tsx)"]}
\NormalTok{        H["Assets (images, animations)"]}
\NormalTok{    end}
    
\NormalTok{    subgraph APIRoutes["API Routes (Next.js)"]}
\NormalTok{        I["Auth API"]}
\NormalTok{        J["Tracker API"]}
\NormalTok{        K["Stress API"]}
\NormalTok{        L["Activities API"]}
\NormalTok{        M["Admin API"]}
\NormalTok{    end}
    
\NormalTok{    \%\% Relations entre composants}
\NormalTok{    F {-}{-}\textgreater{} G}
\NormalTok{    F {-}{-}\textgreater{} H}
\NormalTok{    A {-}{-}\textgreater{} I}
\NormalTok{    A {-}{-}\textgreater{} J}
\NormalTok{    A {-}{-}\textgreater{} K}
\NormalTok{    A {-}{-}\textgreater{} L}
\NormalTok{    A {-}{-}\textgreater{} M}
\end{Highlighting}
\end{Shaded}

\subsubsection{3.2 Design Patterns
utilisés}\label{design-patterns-utilisuxe9s}

\paragraph{3.2.1 Pattern MVC
(Model-View-Controller)}\label{pattern-mvc-model-view-controller}

\begin{Shaded}
\begin{Highlighting}[]
\NormalTok{classDiagram}
\NormalTok{    \%\% Classes avec descriptions détaillées}
\NormalTok{    class Model \{}
\NormalTok{        +schemasPrisma}
\NormalTok{        +définit la structure des données}
\NormalTok{        +gère la persistance}
\NormalTok{    \}}
    
\NormalTok{    class View \{}
\NormalTok{        +composantsReact}
\NormalTok{        +pagesNextjs}
\NormalTok{        +affiche l\textquotesingle{}interface utilisateur}
\NormalTok{        +réagit aux événements utilisateur}
\NormalTok{    \}}
    
\NormalTok{    class Controller \{}
\NormalTok{        +routesAPI}
\NormalTok{        +traiterRequêtes()}
\NormalTok{        +manipulerDonnées()}
\NormalTok{        +gère la logique métier}
\NormalTok{    \}}
    
\NormalTok{    \%\% Relations avec descriptions}
\NormalTok{    Controller {-}{-}\textgreater{} Model : mise à jour}
\NormalTok{    Controller {-}{-}\textgreater{} View : sélectionne}
\NormalTok{    View {-}{-}\textgreater{} Model : observe}
\end{Highlighting}
\end{Shaded}

\paragraph{3.2.2 Pattern Factory - Utilisé dans le jardin
zen}\label{pattern-factory---utilisuxe9-dans-le-jardin-zen}

\begin{Shaded}
\begin{Highlighting}[]
\NormalTok{classDiagram}
\NormalTok{    GardenElement \textless{}|{-}{-} Bassin : "hérite"}
\NormalTok{    GardenElement \textless{}|{-}{-} Stone : "hérite"}
\NormalTok{    GardenElement \textless{}|{-}{-} Lantern : "hérite"}
\NormalTok{    GardenElement \textless{}|{-}{-} Bridge : "hérite"}
\NormalTok{    GardenElementFactory ..\textgreater{} Bassin : "crée"}
\NormalTok{    GardenElementFactory ..\textgreater{} Stone : "crée"}
\NormalTok{    GardenElementFactory ..\textgreater{} Lantern : "crée"}
\NormalTok{    GardenElementFactory ..\textgreater{} Bridge : "crée"}
    
\NormalTok{    class GardenElement \{}
\NormalTok{        \textless{}\textless{}abstract\textgreater{}\textgreater{}}
\NormalTok{        +render() Element}
\NormalTok{        +handleInteraction() void}
\NormalTok{    \}}
    
\NormalTok{    class GardenElementFactory \{}
\NormalTok{        +createElement(type: string) GardenElement}
\NormalTok{    \}}
    
\NormalTok{    class Bassin \{}
\NormalTok{        +render() Element}
\NormalTok{        +handleInteraction() void}
\NormalTok{        +gestionAnimationEau()}
\NormalTok{    \}}
    
\NormalTok{    class Stone \{}
\NormalTok{        +render() Element}
\NormalTok{        +handleInteraction() void}
\NormalTok{        +gestionAnimationPierre()}
\NormalTok{    \}}
    
\NormalTok{    class Lantern \{}
\NormalTok{        +render() Element}
\NormalTok{        +handleInteraction() void}
\NormalTok{        +gestionLumière()}
\NormalTok{    \}}
    
\NormalTok{    class Bridge \{}
\NormalTok{        +render() Element}
\NormalTok{        +handleInteraction() void}
\NormalTok{        +gestionTraversée()}
\NormalTok{    \}}
\end{Highlighting}
\end{Shaded}

\paragraph{3.2.3 Pattern Command - Utilisé pour les interactions dans le
jardin
zen}\label{pattern-command---utilisuxe9-pour-les-interactions-dans-le-jardin-zen}

\begin{Shaded}
\begin{Highlighting}[]
\NormalTok{classDiagram}
\NormalTok{    Command \textless{}|.. SoundCommand : "implémente"}
\NormalTok{    Command \textless{}|.. NavigationCommand : "implémente"}
\NormalTok{    CommandManager o{-}{-}\textgreater{} Command : "utilise"}
    
\NormalTok{    class Command \{}
\NormalTok{        \textless{}\textless{}interface\textgreater{}\textgreater{}}
\NormalTok{        +execute() void}
\NormalTok{        +undo() void}
\NormalTok{    \}}
    
\NormalTok{    class CommandManager \{}
\NormalTok{        {-}history Command[]}
\NormalTok{        +executeCommand(command: Command) void}
\NormalTok{        +undoLastCommand() void}
\NormalTok{        +getCommandHistory() Command[]}
\NormalTok{    \}}
    
\NormalTok{    class SoundCommand \{}
\NormalTok{        {-}receiver AmbientSound}
\NormalTok{        {-}previousState boolean}
\NormalTok{        +execute() void}
\NormalTok{        +undo() void}
\NormalTok{    \}}
    
\NormalTok{    class NavigationCommand \{}
\NormalTok{        {-}receiver Router}
\NormalTok{        {-}destination string}
\NormalTok{        {-}previousPage string}
\NormalTok{        +execute() void}
\NormalTok{        +undo() void}
\NormalTok{    \}}
\end{Highlighting}
\end{Shaded}

\subsubsection{3.3 Modèle Conceptuel de Données
(MCD)}\label{moduxe8le-conceptuel-de-donnuxe9es-mcd}

\begin{Shaded}
\begin{Highlighting}[]
\NormalTok{erDiagram}
\NormalTok{    \%\% Entités principales}
\NormalTok{    USER \{}
\NormalTok{        int id PK "Clé primaire"}
\NormalTok{        string name "Nom d\textquotesingle{}utilisateur"}
\NormalTok{        string email UK "Email unique"}
\NormalTok{        string password "Mot de passe haché"}
\NormalTok{        enum role "USER ou ADMIN"}
\NormalTok{        boolean isActive "Compte actif"}
\NormalTok{        string resetToken "Token de réinitialisation"}
\NormalTok{        datetime resetTokenExpires "Date d\textquotesingle{}expiration du token"}
\NormalTok{        datetime createdAt "Date de création"}
\NormalTok{        datetime updatedAt "Date de mise à jour"}
\NormalTok{    \}}
    
\NormalTok{    STRESSQUESTION \{}
\NormalTok{        int id PK "Clé primaire"}
\NormalTok{        string event UK "Événement stressant"}
\NormalTok{        int points "Points de stress"}
\NormalTok{    \}}
    
\NormalTok{    STRESSRESULT \{}
\NormalTok{        int id PK "Clé primaire"}
\NormalTok{        int userId FK "Référence à l\textquotesingle{}utilisateur"}
\NormalTok{        int totalScore "Score total de stress"}
\NormalTok{        datetime createdAt "Date du test"}
\NormalTok{    \}}
    
\NormalTok{    EMOTIONTYPE \{}
\NormalTok{        int id PK "Clé primaire"}
\NormalTok{        string name UK "Nom de l\textquotesingle{}émotion"}
\NormalTok{        int level "Niveau (1=principale, 2=secondaire)"}
\NormalTok{        int parentId FK "Référence à l\textquotesingle{}émotion parent"}
\NormalTok{        string color "Couleur principale"}
\NormalTok{        string bgColor "Couleur de fond"}
\NormalTok{    \}}
    
\NormalTok{    EMOTION \{}
\NormalTok{        int id PK "Clé primaire"}
\NormalTok{        int userId FK "Référence à l\textquotesingle{}utilisateur"}
\NormalTok{        int emotionId FK "Référence au type d\textquotesingle{}émotion"}
\NormalTok{        datetime date "Date d\textquotesingle{}enregistrement"}
\NormalTok{        string comment "Commentaire"}
\NormalTok{    \}}
    
\NormalTok{    ACTIVITY \{}
\NormalTok{        int id PK "Clé primaire"}
\NormalTok{        string title UK "Titre de l\textquotesingle{}activité"}
\NormalTok{        string description "Description détaillée"}
\NormalTok{        enum category "BIEN\_ETRE, SPORT, ART, AVENTURE"}
\NormalTok{        string duration "Durée estimée"}
\NormalTok{        string level "Niveau de difficulté"}
\NormalTok{        string location "Lieu de pratique"}
\NormalTok{        string equipment "Équipement requis"}
\NormalTok{        boolean isActive "Disponibilité"}
\NormalTok{        datetime createdAt "Date de création"}
\NormalTok{        datetime updatedAt "Date de mise à jour"}
\NormalTok{    \}}
    
\NormalTok{    FAVORITE \{}
\NormalTok{        int id PK "Clé primaire"}
\NormalTok{        int userId FK "Référence à l\textquotesingle{}utilisateur"}
\NormalTok{        int activityId FK "Référence à l\textquotesingle{}activité"}
\NormalTok{        datetime createdAt "Date d\textquotesingle{}ajout"}
\NormalTok{    \}}
    
\NormalTok{    PAGECONTENT \{}
\NormalTok{        int id PK "Clé primaire"}
\NormalTok{        string page UK "Identifiant de page"}
\NormalTok{        string title "Titre de la page"}
\NormalTok{        string content "Contenu HTML"}
\NormalTok{        datetime updatedAt "Date de mise à jour"}
\NormalTok{    \}}
    
\NormalTok{    \%\% Relations}
\NormalTok{    USER ||{-}{-}o\{ STRESSRESULT : "possède"}
\NormalTok{    USER ||{-}{-}o\{ EMOTION : "enregistre"}
\NormalTok{    USER ||{-}{-}o\{ FAVORITE : "possède"}
\NormalTok{    EMOTIONTYPE ||{-}{-}o\{ EMOTION : "est de type"}
\NormalTok{    EMOTIONTYPE ||{-}{-}o\{ EMOTIONTYPE : "est parent de"}
\NormalTok{    ACTIVITY ||{-}{-}o\{ FAVORITE : "est favori"}
\end{Highlighting}
\end{Shaded}

\subsection{4. Gestion des données personnelles et
sensibles}\label{gestion-des-donnuxe9es-personnelles-et-sensibles}

\subsubsection{4.1 Types de données
collectées}\label{types-de-donnuxe9es-collectuxe9es}

\begin{Shaded}
\begin{Highlighting}[]
\NormalTok{flowchart TD}
\NormalTok{    Root[Données personnelles] {-}{-}\textgreater{} ID[Identification]}
\NormalTok{    Root {-}{-}\textgreater{} Health[Santé et bien{-}être]}
    
\NormalTok{    ID {-}{-}\textgreater{} Email[Email]}
\NormalTok{    ID {-}{-}\textgreater{} Password[Mot de passe]}
\NormalTok{    ID {-}{-}\textgreater{} Username[Nom d\textquotesingle{}utilisateur]}
    
\NormalTok{    Health {-}{-}\textgreater{} Stress[Scores de stress]}
\NormalTok{    Health {-}{-}\textgreater{} EmotionHistory[Historique émotions]}
\NormalTok{    Health {-}{-}\textgreater{} Comments[Commentaires émotions]}
\NormalTok{    Health {-}{-}\textgreater{} Preferences[Préférences activités]}
\end{Highlighting}
\end{Shaded}

\subsubsection{4.2 Mesures de protection des
données}\label{mesures-de-protection-des-donnuxe9es}

\begin{Shaded}
\begin{Highlighting}[]
\NormalTok{flowchart TD}
\NormalTok{    \%\% Nœud principal}
\NormalTok{    A[Données personnelles] {-}{-}\textgreater{} B\{Protection\}}
    
\NormalTok{    \%\% Branche sécurité technique}
\NormalTok{    B {-}{-}\textgreater{} C[Sécurité technique]}
\NormalTok{    C {-}{-}\textgreater{} C1[Hachage des mots de passe]}
\NormalTok{    C {-}{-}\textgreater{} C2[Authentification sécurisée]}
\NormalTok{    C {-}{-}\textgreater{} C3[Validation des données]}
\NormalTok{    C {-}{-}\textgreater{} C4[Tokens à durée limitée]}
    
\NormalTok{    \%\% Branche RGPD}
\NormalTok{    B {-}{-}\textgreater{} D[Conformité RGPD]}
\NormalTok{    D {-}{-}\textgreater{} D1[Collecte limitée]}
\NormalTok{    D {-}{-}\textgreater{} D2[Finalités définies]}
\NormalTok{    D {-}{-}\textgreater{} D3[Conservation limitée]}
\NormalTok{    D {-}{-}\textgreater{} D4[Droit à l\textquotesingle{}oubli]}
\NormalTok{    D {-}{-}\textgreater{} D5[Droit d\textquotesingle{}accès]}
\NormalTok{    D {-}{-}\textgreater{} D6[Droit de rectification]}
\end{Highlighting}
\end{Shaded}

\subsubsection{4.3 Architecture de
sécurité}\label{architecture-de-suxe9curituxe9}

\begin{Shaded}
\begin{Highlighting}[]
\NormalTok{flowchart LR}
\NormalTok{    \%\% Acteurs et composants}
\NormalTok{    A([Utilisateur]) {-}{-}\textgreater{}|HTTPS| B[Frontend NextJS]}
\NormalTok{    B {-}{-}\textgreater{}|HTTPS| C[API Routes]}
\NormalTok{    C {-}{-}\textgreater{} D\{Middleware d\textquotesingle{}authentification\}}
\NormalTok{    D {-}{-}\textgreater{}|JWT valide| E[Services métier]}
\NormalTok{    E {-}{-}\textgreater{}|Requêtes sécurisées| F[(Base de données)]}
    
\NormalTok{    \%\% Notes de sécurité}
\NormalTok{    G[CSRF Protection] {-}.{-} B}
\NormalTok{    H[Vérification JWT] {-}.{-} D}
\NormalTok{    I[Données sensibles chiffrées] {-}.{-} F}
\end{Highlighting}
\end{Shaded}

\subsubsection{4.4 Flux de données
sensibles}\label{flux-de-donnuxe9es-sensibles}

\begin{Shaded}
\begin{Highlighting}[]
\NormalTok{sequenceDiagram}
\NormalTok{    \%\% Acteurs et participants}
\NormalTok{    actor User as Utilisateur}
\NormalTok{    participant Frontend as Frontend (Next.js)}
\NormalTok{    participant API as API Routes}
\NormalTok{    participant Auth as Service d\textquotesingle{}authentification}
\NormalTok{    participant DB as Base de données}
    
\NormalTok{    \%\% Connexion avec style}
\NormalTok{    rect rgb(230, 242, 255)}
\NormalTok{    note right of User: Connexion utilisateur}
\NormalTok{    User{-}\textgreater{}\textgreater{}Frontend: Saisie email/mot de passe}
\NormalTok{    Frontend{-}\textgreater{}\textgreater{}API: POST /api/auth/login}
\NormalTok{    API{-}\textgreater{}\textgreater{}Auth: Vérification des identifiants}
\NormalTok{    Auth{-}\textgreater{}\textgreater{}DB: Recherche de l\textquotesingle{}utilisateur}
\NormalTok{    DB{-}{-}\textgreater{}\textgreater{}Auth: Retourne données utilisateur}
\NormalTok{    Auth{-}\textgreater{}\textgreater{}Auth: Vérifie le hachage du mot de passe}
\NormalTok{    Auth{-}{-}\textgreater{}\textgreater{}API: Session utilisateur}
\NormalTok{    API{-}{-}\textgreater{}\textgreater{}Frontend: JWT + Informations utilisateur}
\NormalTok{    Frontend{-}{-}\textgreater{}\textgreater{}User: Redirection vers dashboard}
\NormalTok{    end}
    
\NormalTok{    \%\% Émotions avec style}
\NormalTok{    rect rgb(230, 255, 230)}
\NormalTok{    note right of User: Enregistrement émotion}
\NormalTok{    User{-}\textgreater{}\textgreater{}Frontend: Sélectionne émotion + commentaire}
\NormalTok{    Frontend{-}\textgreater{}\textgreater{}API: POST /api/tracker (avec JWT)}
\NormalTok{    API{-}\textgreater{}\textgreater{}Auth: Vérifie l\textquotesingle{}authentification}
\NormalTok{    Auth{-}{-}\textgreater{}\textgreater{}API: Utilisateur authentifié}
\NormalTok{    API{-}\textgreater{}\textgreater{}DB: Enregistre l\textquotesingle{}émotion}
\NormalTok{    DB{-}{-}\textgreater{}\textgreater{}API: Confirmation}
\NormalTok{    API{-}{-}\textgreater{}\textgreater{}Frontend: Succès}
\NormalTok{    Frontend{-}{-}\textgreater{}\textgreater{}User: Affiche confirmation}
\NormalTok{    end}
\end{Highlighting}
\end{Shaded}

\subsection{5. Récapitulatif des fonctionnalités par
module}\label{ruxe9capitulatif-des-fonctionnalituxe9s-par-module}

\begin{Shaded}
\begin{Highlighting}[]
\NormalTok{graph TD}
\NormalTok{    \%\% Style général}
\NormalTok{    classDef implemented fill:\#d5e8d4,stroke:\#82b366,stroke{-}width:2px}
\NormalTok{    classDef pending fill:\#fff2cc,stroke:\#d6b656,stroke{-}width:2px}
\NormalTok{    classDef moduleStyle fill:\#dae8fc,stroke:\#6c8ebf,stroke{-}width:3px}
    
\NormalTok{    \%\% Modules principaux}
\NormalTok{    Auth[Authentification]:::moduleStyle}
\NormalTok{    Breath[Respiration]:::moduleStyle}
\NormalTok{    Stress[Tracker de stress]:::moduleStyle}
\NormalTok{    Emotion[Tracker d\textquotesingle{}émotions]:::moduleStyle}
\NormalTok{    Activity[Activités]:::moduleStyle}
\NormalTok{    Zen[Jardin zen]:::moduleStyle}
\NormalTok{    Admin[Administration]:::moduleStyle}
    
\NormalTok{    \%\% Fonctionnalités d\textquotesingle{}authentification}
\NormalTok{    Auth {-}{-}\textgreater{} Auth1[Inscription]:::implemented}
\NormalTok{    Auth {-}{-}\textgreater{} Auth2[Connexion]:::implemented}
\NormalTok{    Auth {-}{-}\textgreater{} Auth3[Réinitialisation du mot de passe]:::implemented}
\NormalTok{    Auth {-}{-}\textgreater{} Auth4[Protection des routes]:::implemented}
\NormalTok{    Auth {-}{-}\textgreater{} Auth5[Gestion des rôles]:::implemented}
    
\NormalTok{    \%\% Fonctionnalités de respiration}
\NormalTok{    Breath {-}{-}\textgreater{} Breath1[Exercices prédéfinis]:::implemented}
\NormalTok{    Breath {-}{-}\textgreater{} Breath2[Animation cycle respiratoire]:::implemented}
\NormalTok{    Breath {-}{-}\textgreater{} Breath3[Paramétrage des durées]:::implemented}
\NormalTok{    Breath {-}{-}\textgreater{} Breath4[Contrôles]:::implemented}
\NormalTok{    Breath {-}{-}\textgreater{} Breath5[Historique des sessions]:::pending}
    
\NormalTok{    \%\% Fonctionnalités de stress}
\NormalTok{    Stress {-}{-}\textgreater{} Stress1[Questions avec pondération]:::implemented}
\NormalTok{    Stress {-}{-}\textgreater{} Stress2[Calcul du score]:::implemented}
\NormalTok{    Stress {-}{-}\textgreater{} Stress3[Évaluation du niveau]:::implemented}
\NormalTok{    Stress {-}{-}\textgreater{} Stress4[Historique des tests]:::implemented}
\NormalTok{    Stress {-}{-}\textgreater{} Stress5[Recommandations]:::implemented}
    
\NormalTok{    \%\% Fonctionnalités d\textquotesingle{}émotions}
\NormalTok{    Emotion {-}{-}\textgreater{} Emotion1[Sélection d\textquotesingle{}émotions]:::implemented}
\NormalTok{    Emotion {-}{-}\textgreater{} Emotion2[Ajout de commentaires]:::implemented}
\NormalTok{    Emotion {-}{-}\textgreater{} Emotion3[Visualisation chronologique]:::implemented}
\NormalTok{    Emotion {-}{-}\textgreater{} Emotion4[Suppression d\textquotesingle{}entrées]:::implemented}
\NormalTok{    Emotion {-}{-}\textgreater{} Emotion5[Analyse des tendances]:::pending}
    
\NormalTok{    \%\% Fonctionnalités d\textquotesingle{}activités}
\NormalTok{    Activity {-}{-}\textgreater{} Activity1[Catalogue catégorisé]:::implemented}
\NormalTok{    Activity {-}{-}\textgreater{} Activity2[Filtres de recherche]:::implemented}
\NormalTok{    Activity {-}{-}\textgreater{} Activity3[Système de favoris]:::implemented}
\NormalTok{    Activity {-}{-}\textgreater{} Activity4[Détails des activités]:::implemented}
\NormalTok{    Activity {-}{-}\textgreater{} Activity5[Recommandations personnalisées]:::pending}
    
\NormalTok{    \%\% Fonctionnalités du jardin zen}
\NormalTok{    Zen {-}{-}\textgreater{} Zen1[Interface interactive]:::implemented}
\NormalTok{    Zen {-}{-}\textgreater{} Zen2[Navigation fonctionnalités]:::implemented}
\NormalTok{    Zen {-}{-}\textgreater{} Zen3[Ambiance sonore]:::implemented}
\NormalTok{    Zen {-}{-}\textgreater{} Zen4[Animations visuelles]:::implemented}
\NormalTok{    Zen {-}{-}\textgreater{} Zen5[Éléments personnalisables]:::pending}
    
\NormalTok{    \%\% Fonctionnalités d\textquotesingle{}administration}
\NormalTok{    Admin {-}{-}\textgreater{} Admin1[Gestion des utilisateurs]:::implemented}
\NormalTok{    Admin {-}{-}\textgreater{} Admin2[Gestion des questions de stress]:::implemented}
\NormalTok{    Admin {-}{-}\textgreater{} Admin3[Gestion des types d\textquotesingle{}émotions]:::implemented}
\NormalTok{    Admin {-}{-}\textgreater{} Admin4[Gestion des activités]:::implemented}
\NormalTok{    Admin {-}{-}\textgreater{} Admin5[Gestion du contenu des pages]:::implemented}
\end{Highlighting}
\end{Shaded}

\begin{quote}
\textbf{Légende du diagramme} : Ce diagramme de récapitulatif montre
l’état d’implémentation des fonctionnalités par module. Les rectangles
verts représentent les fonctionnalités déjà implémentées, tandis que les
rectangles jaunes indiquent les fonctionnalités prévues mais pas encore
développées. Chaque module est regroupé avec ses fonctionnalités
spécifiques pour offrir une vision globale de l’avancement du projet.
\end{quote}

\begin{center}\rule{0.5\linewidth}{0.5pt}\end{center}

\end{document}
